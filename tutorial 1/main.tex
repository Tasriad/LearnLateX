%preamble
\documentclass[14pt]{article}
\usepackage{multicol}
\usepackage{multirow}
%title portion
\title{Introduction to Latex}
\author{Tasriad Ahmed Tias}
\date{\today}
%title portion
%preamble

\begin{document}
\maketitle %make title portion visible
\tableofcontents %auto detect contents and generate
\newpage
This is a \LaTeX\ article.
\section{Introduction}
This is the introduction part of the document.\newline
\textbf{This text will be in bold format.}\newline
\textit{This text will be in italic format.}\newline
\underline {This text will be underlined.}\newline
{\Large Large text.}\newline
{\Huge Huge text.}\newline
\emph{This text is emphasized.}
\section{Beginning}
This is the beginning part of the document.
\section{Main}
\subsection{part1}
part 1 of the document.
\subsection{part2}
part 2 of the document.
\subsection{part3}
\subsubsection{part of part3}
this is the sub sub-section of part 3.
\section*{unlisted section}
This section will not show up in the table of contents. Nor will it be numbered.
\newpage
\section{List}
In this section, we shall demonstrate the ordered and unordered list.
\subsection{unordered list}
\begin{itemize}
    \item item1
    \item item2
    \begin{itemize}
        \item nested item1
        \item nested item2
    \end{itemize}
    \item item3
\end{itemize}
\subsection{ordered list}
\begin{enumerate}
    \item item1
    \item item2
    \item item3
    \begin{enumerate}
        \item nested item1
        \item nested item2
    \end{enumerate}
\end{enumerate}
\subsection{descriptive list}
\begin{description}
\item[description1:] about description 1
\item[description2:] about description 2
\end{description}
\section{Table}
We will see tables here.
\subsection{Basic table}
\begin{tabular}{|c|c|c|}
\hline
\textbf{hcell1} & \textbf{hcell2} & \textbf{hcell3}\\
\hline
cell4 & cell5 & cell6\\
\hline
cell7 & cell8 & cell9\\
\hline
\end{tabular}
\subsection{Multi-column table}
\begin{tabular}{|c|c|c|}
    \hline
    cell1 & cell2 & cell3\\
    \hline
    1 & \multicolumn{2}{c|}{Merged}\\
    \hline
    2 & 3 & 4\\
    \hline
\end{tabular}
\subsection{Multi-row table}
\begin{tabular}{|c|c|c|}
    \hline
    cell1 & cell2 & cell3\\
    \hline
    \multirow{2}{*}{Merged} & cell4 & cell5\\
    \cline{2-3}
    & cell6 & cell7\\
    \hline
\end{tabular}
\subsection{Multi-row and Multi-column table}
\begin{tabular}{|c|c|c|c|}
    \hline
    1 & 2 & 3 & 4\\
    \hline
    \multirow{3}{*}{MergedR}
    & \multicolumn{2}{c|}{\multirow{3}{*}{MergedC}} & 5\\
    \cline{4-4}
    & \multicolumn{2}{c|}{} & 6\\
    \cline{4-4}
    & \multicolumn{2}{c|}{} & 7\\
    \hline
\end{tabular}
\section{Mathematical equation}
\subsection{Direct approach}
This is a simple equation $E = mc^2$\newline
This is also a simple equation:\newline 
$a_1x + b_1y = 2$\newline
$a_2x + b_2y = 5$ \newline
$E_{kintetic} = \frac{1}{2}mv^2$\newline
$\sum_{i=0}^n a_i$ \newline
\subsection{Mathematical environment}
\begin{equation}
    x = e^{\frac{a}{b^3}}
\end{equation}

\begin{equation}
    \gamma = \frac{1-f}{f} 
\end{equation}

\begin{equation}
    \epsilon = 8.854E-12
\end{equation}

\begin{equation}
    A \cap B = C
\end{equation}
\begin{equation}
    z= \left(\frac{\frac{x}{y}}{\frac{a}{b}}\right)
\end{equation}
\end{document}